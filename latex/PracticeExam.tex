\documentclass[english,12pt]{amsart}
\RequirePackage{geometry,amsmath,graphicx,babel}
\geometry{verbose,letterpaper,tmargin=1in,bmargin=1in,lmargin=1.25in,rmargin=1.25in,headheight=0.5in,footskip=0.5in}
\setlength{\parskip}{\bigskipamount}
\setlength{\parindent}{0pt}
\newcommand{\head}[1]{\vskip 2.5\baselineskip\begin{center}\underline{\textbf{#1}}\vskip 0.5\baselineskip\end{center}}
\newcommand{\tophead}[1]{\underline{\textbf{#1}}\vskip 0.5\baselineskip}
\usepackage{fancyhdr}
\pagestyle{fancy}
\lhead{\parbox[b][0.9in][t]{0.6\textwidth}{\textsc{Final Exam\\BUSI 721: Foundations of Finance\\Fall 2021\\Professor Kerry Back}}}
\rhead{\vspace*{1cm}\includegraphics[scale=1]{Rice.jpg}\vspace*{-1cm}}
\renewcommand{\headrulewidth}{0.4pt}
%\usepackage[hyphens]{url}

\usepackage{hyperref}

%%%%%%%%%%%%%%%%%%%%%%%%%%%%%%%%%%%%%%%%%%%%%%%%%%%%%%%%%%%%%%%%%%%%%%%%%%%%

\begin{document}
\thispagestyle{fancy}
\vspace*{0.4in}

\vskip 0.4in
 

 \vspace*{0.25in}
 \begin{enumerate}\renewcommand{\labelenumi}{\arabic{enumi}.}
 
 \item \textbf{5 points} \hspace{0.5ex}  You hope to have \$2,000,000 when you retire in 20 years.  You expect to earn 10\% on your investments.  How much must you save each year starting today (i.e., at dates 0, 1, \ldots, 19) to reach your target?
 \item \textbf{5 points} \hspace{0.5ex}  A five-year bond with \$1,000 face value pays coupons of \$30 every six months (the first of ten coupons will be six months from today).  It is selling for \$1,100.  What is its yield (expressed as an annual rate)?
 \item \textbf{20 points} \hspace{0.5ex} You are considering launching a new product.  It requires spending \$40 million on new equipment that would be depreciated straight line to zero over 4 years.  The equipment is not expected to have any salvage value after 4 years.  You would also have to make an upfront  investment of \$5 million in working capital.  Afterwards, net working capital would be 50\% of sales, with recovery at the end of 4 years.  Anticipated sales are \$20 million in year 1, \$25 million in year~2, \$20 million in year 3, and \$15 million in year 4.  Costs of goods sold is projected to be 30\% of sales, and SG\&A expenses are projected to be 10\% of sales.  Your tax rate is 40\%, and your cost of capital is 10\%.  What is the NPV of the project?
 \item \textbf{20 points} \hspace{0.5ex}  In connection with the previous problem, it occurred to you that perhaps you should plan to abandon the project after 3 years, because projected sales are so low in year 4.  You think you could sell the equipment for \$5 million at the end of year 3.  Does the revised project look better or worse than the original?
 \item \textbf{10 points} \hspace{0.5ex}  A company's EBIAT in the past year was \$50 million.  It had capital expenditures of \$40 million and depreciation of \$20 million.  Its net working capital at the end of the past year was \$100 million.  You project that everything (EBIAT, capital expenditures, depreciation, and net working capital) will grow at 4\% forever.  The appropriate discount rate is 10\%.  What is the value of the company?
 \item \textbf{20 points} \hspace{0.5ex}  Stock A has an expected return of 10\% and a standard deviation of 20\%.  Stock B has an expected return of 14\% and a standard deviation of 25\%.  The correlation of the two stock returns is 50\%.  
 \begin{enumerate}
     \item What is the expected return of a portfolio that is 50\% in stock A and 50\% in stock B?
     \item What is the standard deviation of a portfolio that is 50\% in stock A and 50\% in stock B?
     \item Plot the expected returns and standard deviations of all portfolios of stock A and stock B in which the weights on both stocks are nonnegative (no short sales).
     \item Suppose the risk-free rate is 2\%.  What is the maximum Sharpe ratio (tangency) portfolio of the two stocks?  What is its Sharpe ratio?
 \end{enumerate}
 \newpage \thispagestyle{plain}
 \item \textbf{10 points} \hspace{0.5ex} A firm's beta is 1.25.  The market risk premium is 8\%.  The risk-free rate is 2\%.  The value of the firm's equity is \$3 billion, and the value of its debt is \$1 billion.  Its borrowing rate is 5\% and its tax rate is 40\%.
 \begin{enumerate}
 \item What is its cost of equity capital?
 \item What is its weighted average cost of capital?
 \end{enumerate}
 \end{enumerate}
 
 \end{document}
 
 \item \textbf{20 points} \hspace{0.5ex} You are a baker and wish to hedge the price of wheat.  You can hedge with futures or options.
 \begin{enumerate}
 \item If you hedge with futures, should you buy wheat futures contracts or sell them?
 \item If you hedge with options, should you use puts or calls, and should you buy them or sell them?
 \item Compare the hedges you chose in parts (a) and (b) based on the spot price of wheat at the time you need wheat (when the contracts mature).  Under what circumstances will you be happy you chose futures and under what circumstances will you be happy you chose options? 
 \end{enumerate}
 
 \newpage\thispagestyle{plain}
 
 \item \textbf{20 points} \hspace{0.5ex}  On October 9, you decided to buy and some sell some options on Apple that expired on October 30.  You bought a call with a strike of \$114, sold two calls with a strike of \$116, and bought a call with a strike of \$118.  The call prices (premiums) were \$6.25 for the \$114 call, \$5.20 for the \$116 call and \$4.30 for the \$118 call. 
 \begin{enumerate}
 \item If Apple finished at \$120 on October 30, how much did you gain/lose?
 \item Under what circumstances will this trade turn out to be profitable?  
 \item What is the most you can make on the trade and what is the most you can lose (in dollars, not percent)?
 \end{enumerate}
 
 \end{enumerate}
 \end{document}
 
 You bought a crude oil futures contract on day 0 at \$40 per barrel.  Each contract is for 1,000 barrels.  You posted \$4,000 as margin.  
 \begin{enumerate}
 \item Specify how much money will be in your margin account at the end of days 1 through 4 if the futures prices are as specified below. 
 \begin{itemize}
 \item Day 1: \$40.50
 \item Day 2: \$39.50
 \item Day 3: \$39.00
 \item Day 4: \$40.00
 \end{itemize}
 \item You unwind the position by selling the contract on day 5 at \$41 per barrel.  Including margin deposits and debits, what is your total gain or loss on the contract?
 \end{enumerate} 
 \item You sold 100 crude oil futures contracts for December delivery at a futures price of \$40.  You unwound the position by buying 100 contracts in November at a futures=spot price of \$38.  Assuming you sold 100,000 barrels at the spot price of \$38, what is the total price per barrel you earned including the gain or loss on the futures contract?
 \item You produce 2 barrels of gasoline and 1 barrel of heating oil for each 3 barrels of crude oil that you refine.   You want to hedge 100,000 barrels of production in July.  July gasoline futures are at \$1.26 per gallon, July heating oil futures are at \$1.24 per gallon, and July crude oil futures are at \$41.38 per barrel.  A gasoline futures contract is for 42,000 gallons, a heating oil futures contract is for 42,000 gallons, and a crude oil futures contract is for 1,000 barrels (there are 42 gallons in a barrel).
 \begin{enumerate}
 \item What contracts would you buy/sell to fully hedge your 100,000 barrels of production?  What spread would you achieve, per barrel of production?
 \item Suppose spot prices for July delivery turn out to be \$1.13 for gasoline, \$1.11 for heating oil, and \$37.24 for crude oil.
 \begin{enumerate}
 \item If you had not hedged, what spread would you have achieved per barrel of production?
 \item If you had only hedged your crude oil costs and had not hedged your refined product revenue, what spread would you have achieved per barrel of production?
 \end{enumerate}
 \end{enumerate}
 \newpage
 \item You want to hedge the cost of 100,000 barrels of crude oil to be received in January.  The January futures price is \$37.82.  A call option on the January futures with a strike price of \$37.50 is trading at \$2.71.  You buy 100 call options to hedge.
 \begin{enumerate}
 \item Suppose that in December the futures=spot price for January delivery turns out to be \$40.  What is your gain or loss on the options?  What will be your all-in cost of crude, including the option gain or loss?
 \item Answer the same questions assuming that in December the futures=spot price for January delivery turns out to be \$35.
 \end{enumerate}
 \item Consider the same scenario as in the previous exercise.  You decide to buy calls with a strike of \$38.50 that are trading at \$2.23 and to sell puts with a strike of \$36.50 that are trading at \$2.15.
 \begin{enumerate}
 \item What is this type of hedge called?
 \item What is your all-in cost of crude in the following scenarios:
 \begin{enumerate}
 \item In December the futures=spot price for January delivery is \$35.
 \item In December the futures=spot price for January delivery is \$40.
 \item In December the futures=spot price for January delivery is \$45.
 \end{enumerate}
 \end{enumerate}
 \end{enumerate}
 \thispagestyle{plain}
\end{document}

